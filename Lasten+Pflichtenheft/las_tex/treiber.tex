\chapter{Projekttreiber}

\section{Projektziel}

Im Rahmen des Software-Entwicklungs-Projekts {\the\year} soll ein einfach zu bedienendes Client-Server-System zum Spielen von LAMA über ein Netzwerk implementiert werden. Die Benutzeroberfläche soll intuitiv bedienbar sein.

\section{Stakeholders}

\newcounter{sh}\setcounter{sh}{10}

\begin{description}[leftmargin=5em, style=sameline]
	
	\begin{lhp}{sh}{SH}{sh:Spieler}
		\item [Name:] Spieler
		\item [Beschreibung:] Menschliche Spieler.
		\item [Ziele/Aufgaben:] Das Spiel zu spielen.
	\end{lhp}
	
	\begin{lhp}{sh}{SH}{bsh:Spieler}
		\item [Name:] Eltern
		\item [Beschreibung:] Eltern minderjähriger Spieler.
		\item [Ziele/Aufgaben:] Um die Spieler zu kümmern, indem Eltern Spielzeit begrenzen wollen und zugriff auf sensible Inhalte begrenzen.
	\end{lhp}
	
	\begin{lhp}{sh}{SH}{bsh:gesetzgeber}
		\item [Name:] Gesetzgeber
		\item [Beschreibung:] Das Amt für Jugend und Familie.
		\item [Ziele/Aufgaben:] Die Rechte der Spieler zu schützen und zu gewähren, indem er Gesetze erstellt.
	\end{lhp}
	
	\begin{lhp}{sh}{SH}{bsh:investor}
		\item [Name:] Investoren (nur für Beispielzwecken)
		\item [Beschreibung:] Parteien, die das Finanzmittel für die Entwicklung des Systems bereitstellen.
		\item [Ziele/Aufgaben:] Gewinn zu ermitteln, indem das System an Endverbraucher verkauft wird.
	\end{lhp}
	
	\begin{lhp}{sh}{SH}{bsh:betreuer}
		\item [Name:] Betreuer
		\item [Beschreibung:] HiWis, die SEP Projektgruppen betreuen.
		\item [Ziele/Aufgaben:] Das Entwicklungsprozess zu betreuen, zu überwachen und teilweise zu steuern als auch die Arbeit der Projektgruppen abzunehmen sowie den Studenten im Prozess Hilfe zur Verfügung zu stellen. 
	\end{lhp}
	
	\begin{lhp}{sh}{SH}{bsh:prof}
		\item [Name:] apl. Prof. Dr. Achim Ebert
		\item [Beschreibung:] Organisiert Veranstaltung(SEP).
		\item [Ziele/Aufgaben:] Beauftragt Betreuer und sorgt für eine bessere Annäherung der Studenten an Softwareprojekte in Unternehmen, um sie auf das Berufsleben vorzubereiten.
	\end{lhp}
		
\end{description}

\section{Aktuelle Lage}

Aktuell wird das Spiel so gespielt, dass eine physische Version des Spiels erforderlich ist. Das Problem dabei ist, dass die Karten beschädigt werden oder verloren gehen können. Auch aufgrund der Corona-Pandemie ist es für eine Gruppe von Freunden ziemlich riskant, physisch an einem Ort anwesend zu sein, um das Spiel gemeinsam zu spielen.  Das Projekt ermöglicht es den Spielern, das Spiel mit anderen Spielern, unabhängig von Zeit und Ort dieser zu spielen.  Eltern müssen sich keine Sorgen um die Gesundheit ihrer Kinder machen, da alles online ist. Darüber hinaus haben sie die Möglichkeit, ihre Spielzeit zu begrenzen und sicherzustellen, dass sie keine sensiblen Inhalte sehen können.